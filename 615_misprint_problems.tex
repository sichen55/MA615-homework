\documentclass[]{article}
\usepackage{lmodern}
\usepackage{amssymb,amsmath}
\usepackage{ifxetex,ifluatex}
\usepackage{fixltx2e} % provides \textsubscript
\ifnum 0\ifxetex 1\fi\ifluatex 1\fi=0 % if pdftex
  \usepackage[T1]{fontenc}
  \usepackage[utf8]{inputenc}
\else % if luatex or xelatex
  \ifxetex
    \usepackage{mathspec}
  \else
    \usepackage{fontspec}
  \fi
  \defaultfontfeatures{Ligatures=TeX,Scale=MatchLowercase}
\fi
% use upquote if available, for straight quotes in verbatim environments
\IfFileExists{upquote.sty}{\usepackage{upquote}}{}
% use microtype if available
\IfFileExists{microtype.sty}{%
\usepackage{microtype}
\UseMicrotypeSet[protrusion]{basicmath} % disable protrusion for tt fonts
}{}
\usepackage[margin=1in]{geometry}
\usepackage{hyperref}
\hypersetup{unicode=true,
            pdftitle={MA615 The problem of misprint},
            pdfauthor={Si Chen},
            pdfborder={0 0 0},
            breaklinks=true}
\urlstyle{same}  % don't use monospace font for urls
\usepackage{color}
\usepackage{fancyvrb}
\newcommand{\VerbBar}{|}
\newcommand{\VERB}{\Verb[commandchars=\\\{\}]}
\DefineVerbatimEnvironment{Highlighting}{Verbatim}{commandchars=\\\{\}}
% Add ',fontsize=\small' for more characters per line
\usepackage{framed}
\definecolor{shadecolor}{RGB}{248,248,248}
\newenvironment{Shaded}{\begin{snugshade}}{\end{snugshade}}
\newcommand{\KeywordTok}[1]{\textcolor[rgb]{0.13,0.29,0.53}{\textbf{#1}}}
\newcommand{\DataTypeTok}[1]{\textcolor[rgb]{0.13,0.29,0.53}{#1}}
\newcommand{\DecValTok}[1]{\textcolor[rgb]{0.00,0.00,0.81}{#1}}
\newcommand{\BaseNTok}[1]{\textcolor[rgb]{0.00,0.00,0.81}{#1}}
\newcommand{\FloatTok}[1]{\textcolor[rgb]{0.00,0.00,0.81}{#1}}
\newcommand{\ConstantTok}[1]{\textcolor[rgb]{0.00,0.00,0.00}{#1}}
\newcommand{\CharTok}[1]{\textcolor[rgb]{0.31,0.60,0.02}{#1}}
\newcommand{\SpecialCharTok}[1]{\textcolor[rgb]{0.00,0.00,0.00}{#1}}
\newcommand{\StringTok}[1]{\textcolor[rgb]{0.31,0.60,0.02}{#1}}
\newcommand{\VerbatimStringTok}[1]{\textcolor[rgb]{0.31,0.60,0.02}{#1}}
\newcommand{\SpecialStringTok}[1]{\textcolor[rgb]{0.31,0.60,0.02}{#1}}
\newcommand{\ImportTok}[1]{#1}
\newcommand{\CommentTok}[1]{\textcolor[rgb]{0.56,0.35,0.01}{\textit{#1}}}
\newcommand{\DocumentationTok}[1]{\textcolor[rgb]{0.56,0.35,0.01}{\textbf{\textit{#1}}}}
\newcommand{\AnnotationTok}[1]{\textcolor[rgb]{0.56,0.35,0.01}{\textbf{\textit{#1}}}}
\newcommand{\CommentVarTok}[1]{\textcolor[rgb]{0.56,0.35,0.01}{\textbf{\textit{#1}}}}
\newcommand{\OtherTok}[1]{\textcolor[rgb]{0.56,0.35,0.01}{#1}}
\newcommand{\FunctionTok}[1]{\textcolor[rgb]{0.00,0.00,0.00}{#1}}
\newcommand{\VariableTok}[1]{\textcolor[rgb]{0.00,0.00,0.00}{#1}}
\newcommand{\ControlFlowTok}[1]{\textcolor[rgb]{0.13,0.29,0.53}{\textbf{#1}}}
\newcommand{\OperatorTok}[1]{\textcolor[rgb]{0.81,0.36,0.00}{\textbf{#1}}}
\newcommand{\BuiltInTok}[1]{#1}
\newcommand{\ExtensionTok}[1]{#1}
\newcommand{\PreprocessorTok}[1]{\textcolor[rgb]{0.56,0.35,0.01}{\textit{#1}}}
\newcommand{\AttributeTok}[1]{\textcolor[rgb]{0.77,0.63,0.00}{#1}}
\newcommand{\RegionMarkerTok}[1]{#1}
\newcommand{\InformationTok}[1]{\textcolor[rgb]{0.56,0.35,0.01}{\textbf{\textit{#1}}}}
\newcommand{\WarningTok}[1]{\textcolor[rgb]{0.56,0.35,0.01}{\textbf{\textit{#1}}}}
\newcommand{\AlertTok}[1]{\textcolor[rgb]{0.94,0.16,0.16}{#1}}
\newcommand{\ErrorTok}[1]{\textcolor[rgb]{0.64,0.00,0.00}{\textbf{#1}}}
\newcommand{\NormalTok}[1]{#1}
\usepackage{graphicx,grffile}
\makeatletter
\def\maxwidth{\ifdim\Gin@nat@width>\linewidth\linewidth\else\Gin@nat@width\fi}
\def\maxheight{\ifdim\Gin@nat@height>\textheight\textheight\else\Gin@nat@height\fi}
\makeatother
% Scale images if necessary, so that they will not overflow the page
% margins by default, and it is still possible to overwrite the defaults
% using explicit options in \includegraphics[width, height, ...]{}
\setkeys{Gin}{width=\maxwidth,height=\maxheight,keepaspectratio}
\IfFileExists{parskip.sty}{%
\usepackage{parskip}
}{% else
\setlength{\parindent}{0pt}
\setlength{\parskip}{6pt plus 2pt minus 1pt}
}
\setlength{\emergencystretch}{3em}  % prevent overfull lines
\providecommand{\tightlist}{%
  \setlength{\itemsep}{0pt}\setlength{\parskip}{0pt}}
\setcounter{secnumdepth}{0}
% Redefines (sub)paragraphs to behave more like sections
\ifx\paragraph\undefined\else
\let\oldparagraph\paragraph
\renewcommand{\paragraph}[1]{\oldparagraph{#1}\mbox{}}
\fi
\ifx\subparagraph\undefined\else
\let\oldsubparagraph\subparagraph
\renewcommand{\subparagraph}[1]{\oldsubparagraph{#1}\mbox{}}
\fi

%%% Use protect on footnotes to avoid problems with footnotes in titles
\let\rmarkdownfootnote\footnote%
\def\footnote{\protect\rmarkdownfootnote}

%%% Change title format to be more compact
\usepackage{titling}

% Create subtitle command for use in maketitle
\newcommand{\subtitle}[1]{
  \posttitle{
    \begin{center}\large#1\end{center}
    }
}

\setlength{\droptitle}{-2em}

  \title{MA615 The problem of misprint}
    \pretitle{\vspace{\droptitle}\centering\huge}
  \posttitle{\par}
    \author{Si Chen}
    \preauthor{\centering\large\emph}
  \postauthor{\par}
      \predate{\centering\large\emph}
  \postdate{\par}
    \date{September 16, 2018}

\usepackage{booktabs}
\usepackage{longtable}
\usepackage{array}
\usepackage{multirow}
\usepackage[table]{xcolor}
\usepackage{wrapfig}
\usepackage{float}
\usepackage{colortbl}
\usepackage{pdflscape}
\usepackage{tabu}
\usepackage{threeparttable}
\usepackage{threeparttablex}
\usepackage[normalem]{ulem}
\usepackage{makecell}

\begin{document}
\maketitle

\begin{Shaded}
\begin{Highlighting}[]
\KeywordTok{library}\NormalTok{(knitr)}
\KeywordTok{library}\NormalTok{(kableExtra)}

\CommentTok{# suoppose there are usually no more than 6 misprints in one page for the book}

\NormalTok{m <-}\StringTok{ }\KeywordTok{c}\NormalTok{(}\DecValTok{0}\OperatorTok{:}\DecValTok{6}\NormalTok{)}

\CommentTok{# for each page, it may have 0,1,2,3,4,5,6 misprints.}

\NormalTok{misprint <-}\StringTok{ }\KeywordTok{ppois}\NormalTok{(m,}\DecValTok{2}\NormalTok{,}\DataTypeTok{lower.tail =} \OtherTok{FALSE}\NormalTok{)}
\NormalTok{misprint}
\end{Highlighting}
\end{Shaded}

\begin{verbatim}
## [1] 0.864664717 0.593994150 0.323323584 0.142876540 0.052653017 0.016563608
## [7] 0.004533806
\end{verbatim}

\begin{Shaded}
\begin{Highlighting}[]
\CommentTok{# use binomial function to count for every 10 page:10,20,30,40,50}
\CommentTok{# define each page range}

\NormalTok{errorpage10 <-}\StringTok{ }\KeywordTok{dbinom}\NormalTok{(}\DecValTok{1}\NormalTok{,}\DecValTok{10}\NormalTok{,misprint)}
\NormalTok{errorpage20 <-}\StringTok{ }\KeywordTok{dbinom}\NormalTok{(}\DecValTok{11}\NormalTok{,}\DecValTok{20}\NormalTok{,misprint)}
\NormalTok{errorpage30 <-}\StringTok{ }\KeywordTok{dbinom}\NormalTok{(}\DecValTok{21}\NormalTok{,}\DecValTok{30}\NormalTok{,misprint)}
\NormalTok{errorpage40 <-}\StringTok{ }\KeywordTok{dbinom}\NormalTok{(}\DecValTok{31}\NormalTok{,}\DecValTok{40}\NormalTok{,misprint)}
\NormalTok{errorpage50 <-}\StringTok{ }\KeywordTok{dbinom}\NormalTok{(}\DecValTok{41}\NormalTok{,}\DecValTok{50}\NormalTok{,misprint)}

\CommentTok{# combine five 10-pages ranges}
\NormalTok{errorpage <-}\StringTok{ }\KeywordTok{cbind}\NormalTok{(errorpage10, errorpage20, errorpage30, errorpage40, errorpage50)}

\CommentTok{# set the table names}
\KeywordTok{colnames}\NormalTok{(errorpage) <-}\StringTok{ }\KeywordTok{c}\NormalTok{(}\StringTok{"page 1-10"}\NormalTok{,}\StringTok{"page 11-20"}\NormalTok{,}\StringTok{"page 21-30"}\NormalTok{,}\StringTok{"page 31-40"}\NormalTok{,}\StringTok{"page 41-50"}\NormalTok{)}
\KeywordTok{rownames}\NormalTok{(errorpage) <-}\StringTok{ }\KeywordTok{c}\NormalTok{(}\StringTok{"0 error"}\NormalTok{,}\StringTok{"1 error"}\NormalTok{,}\StringTok{"2 errors"}\NormalTok{,}\StringTok{"3 errors"}\NormalTok{,}\StringTok{"4 errors"}\NormalTok{,}\StringTok{"5errors"}\NormalTok{,}\StringTok{"6 errors"}\NormalTok{)}

\CommentTok{# construct table}
\NormalTok{Errorpage <-}\StringTok{ }\KeywordTok{kable}\NormalTok{(errorpage, }\StringTok{"latex"}\NormalTok{)}

\CommentTok{# table styling}
\KeywordTok{kable_styling}\NormalTok{(Errorpage, }\DataTypeTok{bootstrap_options =} \StringTok{"striped"}\NormalTok{, }\DataTypeTok{font_size=}\DecValTok{16}\NormalTok{ )}
\end{Highlighting}
\end{Shaded}

\begin{table}[H]
\centering\begingroup\fontsize{16}{18}\selectfont

\begin{tabular}{l|r|r|r|r|r}
\hline
  & page 1-10 & page 11-20 & page 21-30 & page 31-40 & page 41-50\\
\hline
0 error & 0.0000001 & 0.0005167 & 0.0102815 & 0.0459028 & 0.0982516\\
\hline
1 error & 0.0017806 & 0.1635311 & 0.0761674 & 0.0079597 & 0.0003988\\
\hline
2 errors & 0.0961760 & 0.0201672 & 0.0000214 & 0.0000000 & 0.0000000\\
\hline
3 errors & 0.3567396 & 0.0000212 & 0.0000000 & 0.0000000 & 0.0000000\\
\hline
4 errors & 0.3235973 & 0.0000000 & 0.0000000 & 0.0000000 & 0.0000000\\
\hline
5errors & 0.1425186 & 0.0000000 & 0.0000000 & 0.0000000 & 0.0000000\\
\hline
6 errors & 0.0435213 & 0.0000000 & 0.0000000 & 0.0000000 & 0.0000000\\
\hline
\end{tabular}\endgroup{}
\end{table}

\begin{Shaded}
\begin{Highlighting}[]
\CommentTok{# insert pic}
\CommentTok{# pic URL:https://github.com/sichen55/MA615-homework.git}
\NormalTok{knitr}\OperatorTok{::}\KeywordTok{include_graphics}\NormalTok{(}\StringTok{'https://github.com/sichen55/MA615-homework/blob/master/books.jpg'}\NormalTok{)}
\end{Highlighting}
\end{Shaded}

\includegraphics{https://github.com/sichen55/MA615-homework/blob/master/books.jpg}

\begin{Shaded}
\begin{Highlighting}[]
\CommentTok{#finished}
\end{Highlighting}
\end{Shaded}

Note: For every page, use possion distribution to calculate the
possibility of making certain number of mistakes.

Use page range by every 10 pages, otherwise if we list every page, we
may need to use loop.

Use binomial function to find the error number in each page, where the
probability found in step 1 is the p value in binomial equation.

set and style the table to expected structure.


\end{document}
